\section{Budget and sustainability}
\subsection{Project budget}
In this section, an estimation of the cost of the project is presented, taking into account the previously mentioned resources.

\subsubsection{Hardware budget}

\autoref{table:res-hw}
shows the hardware resources budget needed in order to develop and test the project.

\begin{table}[h!]
\centering
\begin{tabular}{ |c|c|c|c|c| } 
    \hline
    Product & Price & Units & Useful life & Amortization \\
    \hline
    \hline
    Apple MacBook Pro & 2000€ & 1 & 8 years & 250€ \\ 
    \hline
    Prusa i3 MK3 3D Printer & 769€ & 1 & 4 years & 192.25€ \\ 
    \hline
    \hline
    Total & 2769€ & & & 442.25€ \\
    \hline
\end{tabular}
\caption{Hardware budget}\label{table:res-hw}
\end{table}

\subsubsection{Software budget}

\autoref{table:res-sw}
shows the software resources budget needed in order to develop and test the project.

\begin{table}[h!]
\centering
\begin{tabular}{ |c|c|c|c| } 
    \hline
    Product & Price & Useful life & Amortization \\
    \hline
    \hline
    Git & 0€ & & 0€ \\ 
    \hline
    GitHub & 0€ & & 0€ \\ 
    \hline
    macOS High Sierra & 0€ & & 0€ \\ 
    \hline
    Google Chrome browser & 0€ & & 0€ \\ 
    \hline
    Gantter & 0€ & & 0€ \\ 
    \hline
    Slic3r & 0€ & & 0€ \\ 
    \hline
    Slic3r PE & 0€ & & 0€ \\ 
    \hline
    Ultimaker Cura & 0€ & & 0€ \\ 
    \hline
    \LaTeX & 0€ & & 0€ \\ 
    \hline
    webpack & 0€ & & 0€ \\ 
    \hline
    WebStorm & 129€ & 1 year & 129€ \\ 
    \hline
    \hline
    Total & 129€ & & 129€ \\
    \hline
\end{tabular}
\caption{Software budget}\label{table:res-sw}
\end{table}


\subsubsection{Human resources budget}

\autoref{table:res-hres}
shows the human resources budget needed in order to develop and test the project.

\begin{table}[h!]
\centering
\begin{tabular}{ |c|c|c|c| } 
    \hline
    Role & Price/h & Hours & Cost \\
    \hline
    \hline
    Project manager & 50€ & 50 & 2,500€ \\
    \hline
    Software developer & 30€ & 300 & 9,000€ \\
    \hline
    Tester & 30€ & 100 & 3,000€ \\
    \hline
    \hline
    Total & & & 14,500€ \\
    \hline
\end{tabular}
\caption{Human resources budget}\label{table:res-hres}
\end{table}


Since the modules of the project are mostly independent, and equally important, the different roles are distributed evenly across all tasks.

\subsubsection{Unexpected costs}
\autoref{table:res-extra} shows a generous budget margin for problems or deviations that may arise. This extra budget is very unlikely to be used because of the robust planning of the previous deliverable.


\begin{table}[h!]
\centering
\begin{tabular}{ |c|c|c|c| } 
    \hline
    Role & Price/h & Hours & Cost \\
    \hline
    \hline
    Project manager & 50€ & 10 & 500€ \\
    \hline
    Software developer & 30€ & 20 & 600€ \\
    \hline
    Tester & 30€ & 10 & 300€ \\
    \hline
    \hline
    Total & & & 1,400€ \\
    \hline
\end{tabular}
\caption{Extra human resources budget}\label{table:res-extra}
\end{table}


\subsubsection{Indirect costs}

\autoref{table:res-indirect-costs}
shows the indirect costs related to developing and testing the project.

For this project we will fix the price of electricity at 0.12€/kWh. A 3D printer consumes about 0.2kWh, a laptop about 0.06kWh, and the rest (router, lights) about 0.03kWh. We will assume everything is running during the whole development of the project, except for the 3D printer, which will only be needed in the latest stages of the project, and we will consider it is used only the last 50\% of the project. This brings us at a total electricity consumption of about 100kWh.


\begin{table}[h!]
\centering
\begin{tabular}{ |c|c|c|c| } 
    \hline
    Product & Price & Amount & Cost \\
    \hline
    \hline
    Electricity & 0.12€/kWh & 100kWh & 12€ \\
    \hline
    Internet & 30€/month & 4 months & 120€ \\
    \hline
    Coffee and snacks & 10€/week & 16 weeks & 160€ \\
    \hline
    \hline
    Total & & & 292€ \\
    \hline
\end{tabular}
\caption{Indirect costs}\label{table:res-indirect-costs}
\end{table}

\subsubsection{Total budget}
We will calculate the total budget by adding all previous budgets together. This is shown in \autoref{table:res-total-costs}.

\begin{table}[h!]
\centering
\begin{tabular}{ |c|c| } 
    \hline
    Concept & Estimated cost \\
    \hline
    \hline
    Hardware & 442.25€ \\
    \hline
    Software & 129€ \\
    \hline
    Human resources & 14,500€ \\
    \hline
    Unexpected costs & 1,400€ \\
    \hline
    Indirect costs & 292€ \\
    \hline
    \hline
    Total & 16,763.25€ \\
    \hline
\end{tabular}
\caption{Total budget}\label{table:res-total-costs}
\end{table}

\subsubsection{Budget management}
It is unlikely that we will exceed our budget, since our project planning section was quite robust. We have already accounted for a generous error margin to fix bugs or compensate for delays in the development of complex algorithms. It might still be possible to need extra resources or hire an additional software developer as a consultant. We have also taken this into account when creating our budget. We are being extremely cautious about human resources budget allocation since this is the most expensive part of it.

As we discussed earlier, it is possible to break the 3D printer while doing our testing. In this case we will work on a gcode simulator that comes with most existing slicers. This will add nothing to our budget, and we will continue the project in a theoretical way, without actual 3D printed tests or examples.



\subsection{Sustainability report}
In this section we analyze the sustainability of the project. Its impact is analyzed in its three dimensions: social, economical and environmental.


\subsubsection{Social impact}
On a personal level, this project allows me to push my limits on development and testing, as well as to deduce complex algorithms and development methods. It will help me know my limits and capabilities and provide me with a fun time to learn and experiment.

To the end user, this project does not have a significant social impact, since it is only a tool to convert solid models to 3D printer instructions.


\subsubsection{Economical impact}
We have already discussed and quantified the budget of this project in terms of hardware, software, and human resources. 

We try to use mostly free software, and our product will also be a free software tool. Existing free software slicers exist already, so this project will not make a big economical impact on the end user.


\subsubsection{Environmental impact}
We will be using 100kWh of electricity while developing and testing the project, which is equivalent to approximately 28.3 kg of CO${_2}$. This might seem like a lot until we add our team of humans, who breathe during development, which accounts for approximately 50 more kg of CO${_2}$, for a total of 78.3 kg \cite{co2-breathing}.


We will be using Polylactic Acid (PLA) as our only plastic during the testing phase. It is a bioplastic material, biodegradable and reduces CO$_{2}$ \cite{pla-co2, env-impact-3dp}. This is one of the best materials to use when 3D printing and we choose this for our testing because of its nice environmental properties.






